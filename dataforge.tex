\chapter{Using Data-Forge}\label{s:dataforge}

FIXME: write this chapter

-   Inspired by Python's Pandas library
    -   But should be familiar to users of R dataframes
-   DataForge creates DataFrames with named columns
-   Each column is a Series
-   Key features:
    -   Dataframes are immutable: operations produce new dataframes
    -   Method chaining
    -   Lazy evaluation: calculations aren't done until they are needed
-   Tidy data
    -   Each column is a variable
    -   Each row is an observation
    
\section{Basic Operations}\label{s:dataforge-basics}

-   Load the library and create a dataframe from a list of objects
    -   Each objects must have the same keys

\begin{minted}{js}
const DF = require('data-forge')

const fromObjects = new DF.DataFrame([
    {ones: 1, tens: 10},
    {ones: 2, tens: 20},
    {ones: 3, tens: 30}
])
console.log('fromObjects:\n', fromObjects)
\end{minted}

\begin{minted}{text}
fromObjects:
 DataFrame {
  configFn: null,
  content:
   { index: CountIterable {},
     values: [ [Object], [Object], [Object] ],
     pairs: MultiIterable { iterables: [Array] },
     isBaked: true,
     columnNames: [ 'ones', 'tens' ] } }
\end{minted}

-   Convert to array to see values

\begin{minted}{js}
console.log('fromObjects as array:\n', fromObjects.toArray())
\end{minted}

\begin{minted}{text}
fromObjects as array:
 [ { ones: 1, tens: 10 },
  { ones: 2, tens: 20 },
  { ones: 3, tens: 30 } ]
\end{minted}

-   Can instead create from a spec of column names and row values

\begin{minted}{js}
const fromSpec = new DF.DataFrame({
  columnNames: ['ones', 'tens'],
  rows: [
    [4, 40],
    [5, 50],
    [6, 60]
  ]
})
console.log('fromSpec as array:\n', fromSpec.toArray())
\end{minted}

\begin{minted}{text}
fromSpec as array:
 [ { ones: 4, tens: 40 },
  { ones: 5, tens: 50 },
  { ones: 6, tens: 60 } ]
\end{minted}

-   But will usually get data from a source such as a file
    -   Data-Forge has a companion file I/O library
    -   However you get the characters, it can translate CSV and JSON

\begin{minted}{js}
const text = `ones,tens
7,70
8,80
9,90`
const fromText = DF.fromCSV(text)
console.log('fromText as array:\n', fromText.toArray())
\end{minted}

\begin{minted}{text}
fromText as array:
 [ { ones: '7', tens: '70' },
  { ones: '8', tens: '80' },
  { ones: '9', tens: '90' } ]
\end{minted}

-   However we create it, we can get the column names:

\begin{minted}{js}
console.log(data.getColumnNames())
\end{minted}

\begin{minted}{text}
[ 'ones', 'tens' ]
\end{minted}

-   Or get the values as an array of arrays (rather than as objects):

\begin{minted}{js}
console.log(data.toRows())
\end{minted}

\begin{minted}{text}
[ [ 1, 10 ], [ 2, 20 ], [ 3, 30 ] ]
\end{minted}

-   Or process the rows using either a \texttt{for} loop:

\begin{minted}{js}
for (let row of data) {
  console.log(row)
}
\end{minted}

\begin{minted}{text}
{ ones: 1, tens: 10 }
{ ones: 2, tens: 20 }
{ ones: 3, tens: 30 }
\end{minted}

-   Or \texttt{forEach}:

\begin{minted}{js}
data.forEach(row => {
  console.log(row)
})
\end{minted}

\begin{minted}{text}
{ ones: 1, tens: 10 }
{ ones: 2, tens: 20 }
{ ones: 3, tens: 30 }
\end{minted}

\section{Doing Calculations}\label{s:dataforge-calc}

-   Add a series to a dataframe as a new column
    -   Does \emph{not} modify the existing dataframe

\begin{minted}{js}
const double_oh = new DF.Series([100, 200, 300])

const withHundreds = data.withSeries({hundreds: double_oh})
console.log(withHundreds.toArray())
\end{minted}

\begin{minted}{text}
[ { ones: 1, tens: 10, hundreds: 100 },
  { ones: 2, tens: 20, hundreds: 200 },
  { ones: 3, tens: 30, hundreds: 300 } ]
\end{minted}

\figpdf{figures/FIXME.png}{Everything is a Pipeline}{f:dataforge-pipeline} % show pipeline with new dataframe being created

-   More common to create new columns from existing values

\begin{minted}{js}
const sumsAndProducts = data.generateSeries({
  sum: row => row.ones + row.tens,
  product: row => row.ones * row.tens
})
console.log(sumsAndProducts.toArray())
\end{minted}

\begin{minted}{text}
[ { ones: 1, tens: 10, sum: 11, product: 10 },
  { ones: 2, tens: 20, sum: 22, product: 40 },
  { ones: 3, tens: 30, sum: 33, product: 90 } ]
\end{minted}

-   Can also get rid of columns entirely

\begin{minted}{js}
const justResults = sumsAndProducts.dropSeries(["ones", "tens"])
console.log(justResults.toArray())
\end{minted}

\begin{minted}{text}
[ { sum: 11, product: 10 },
  { sum: 22, product: 40 },
  { sum: 33, product: 90 } ]
\end{minted}

\figpdf{figures/FIXME.png}{A More Complicated Pipeline}{f:dataforge-another-pipeline} % show pipeline with new dataframe being created

-   Often want to sort data for display
-   Raw data:

\begin{minted}{js}
const fs = require('fs')
const DF = require('data-forge')

const text = fs.readFileSync(process.argv[2], 'utf-8')
const colors = DF.fromCSV(text)
console.log(colors.toArray())
\end{minted}

\begin{minted}{text}
[ { name: 'maroon', red: '128', green: '0', blue: '0' },
  { name: 'lime', red: '0', green: '255', blue: '0' },
  { name: 'navy', red: '0', green: '0', blue: '128' },
  { name: 'yellow', red: '255', green: '255', blue: '0' },
  { name: 'fuchsia', red: '255', green: '0', blue: '255' },
  { name: 'aqua', red: '0', green: '255', blue: '255' } ]
\end{minted}

-   Sorted data:

\begin{minted}{js}
const sorted = colors.orderBy(row => row.name)
console.log(sorted.toArray())
\end{minted}

\begin{minted}{text}
[ { name: 'aqua', red: '0', green: '255', blue: '255' },
  { name: 'fuchsia', red: '255', green: '0', blue: '255' },
  { name: 'lime', red: '0', green: '255', blue: '0' },
  { name: 'maroon', red: '128', green: '0', blue: '0' },
  { name: 'navy', red: '0', green: '0', blue: '128' },
  { name: 'yellow', red: '255', green: '255', blue: '0' } ]
\end{minted}

-   Use \texttt{thenBy} to sub-sort by another column

\begin{minted}{js}
const doubleSorted = colors
      .orderBy(row => row.green)
      .thenBy(row => row.blue)
      .dropSeries(['name', 'red'])
console.log(doubleSorted.toArray())
\end{minted}

\begin{minted}{text}
[ { green: '0', blue: '0' },
  { green: '0', blue: '128' },
  { green: '0', blue: '255' },
  { green: '255', blue: '0' },
  { green: '255', blue: '0' },
  { green: '255', blue: '255' } ]
\end{minted}

\figpdf{figures/FIXME.png}{Sorting and Sub-sorting}{f:dataforge-sorting} % show sorting and sub-sorting as diagram

-   Removing duplicates is often combined with sorting for display purposes

\begin{minted}{js}
const notTheSame = colors.distinct(row => row.red)
console.log(notTheSame.toArray())
\end{minted}

\begin{minted}{text}
[ { name: 'maroon', red: '128', green: '0', blue: '0' },
  { name: 'lime', red: '0', green: '255', blue: '0' },
  { name: 'yellow', red: '255', green: '255', blue: '0' } ]
\end{minted}

-   This is trickier than it appears
    -   Each row has a distinct \texttt{red} value,
        but Data-Forge gets to decide which row is kept
-   What's more surprising is that multi-column comparison doesn't work

\begin{minted}{js}
const multiColumn = colors
      .distinct(row => [row.red, row.green])
      .orderBy(row => row.red)
      .thenBy(row => row.green)
console.log(multiColumn.toArray())
\end{minted}

\begin{minted}{text}
[ { name: 'navy', red: '0', green: '0', blue: '128' },
  { name: 'lime', red: '0', green: '255', blue: '0' },
  { name: 'aqua', red: '0', green: '255', blue: '255' },
  { name: 'maroon', red: '128', green: '0', blue: '0' },
  { name: 'fuchsia', red: '255', green: '0', blue: '255' },
  { name: 'yellow', red: '255', green: '255', blue: '0' } ]
\end{minted}

-   Because arrays are only equal if they're the same object
    \texttt{[0] === [0]} is \texttt{false}

\section{Subsets}\label{s:dataforge-subset}

-   Often want to work with subsets of data
    -   And get data into the right format using \texttt{parseDates}, \texttt{parseFloats}, etc.

\begin{minted}{js}
const fs = require('fs')
const DF = require('data-forge')

const text = fs.readFileSync('earthquakes.csv', 'utf-8')
const earthquakes = DF
      .fromCSV(text)
      .parseDates('Time')
      .parseFloats(['Latitude', 'Longitude', 'Depth_Km', 'Magnitude'])
console.log('Data has', earthquakes.count(), 'rows')
\end{minted}

\begin{minted}{text}
Data has 798 rows
\end{minted}

-   Selecting by position using \texttt{head} and \texttt{tail}
    -   Named after classic Unix commands

\begin{minted}{js}
console.log(earthquakes.head(3).toArray())
\end{minted}

\begin{minted}{text}
[ { Time: 2016-08-24T07:36:32.000Z,
    Latitude: 42.6983,
    Longitude: 13.2335,
    Depth_Km: 8.1,
    Magnitude: 6 },
  { Time: 2016-08-24T07:37:26.580Z,
    Latitude: 42.7123,
    Longitude: 13.2533,
    Depth_Km: 9,
    Magnitude: 4.5 },
  { Time: 2016-08-24T07:40:46.590Z,
    Latitude: 42.7647,
    Longitude: 13.1723,
    Depth_Km: 9.7,
    Magnitude: 3.8 } ]
\end{minted}

\begin{minted}{js}
console.log(earthquakes.tail(3).toArray())
\end{minted}

\begin{minted}{text}
[ { Time: 2016-08-26T10:09:45.380Z,
    Latitude: 42.6953,
    Longitude: 13.2363,
    Depth_Km: 9.5,
    Magnitude: 2.3 },
  { Time: 2016-08-26T10:11:55.960Z,
    Latitude: 42.6163,
    Longitude: 13.2985,
    Depth_Km: 11,
    Magnitude: 2.1 },
  { Time: 2016-08-26T10:21:09.870Z,
    Latitude: 42.6153,
    Longitude: 13.2952,
    Depth_Km: 7.5,
    Magnitude: 3 } ]
\end{minted}

-   Or skip a few rows and then take some

\begin{minted}{js}
console.log(earthquakes.skip(10).take(3).toArray())
\end{minted}

\begin{minted}{text}
[ { Time: 2016-08-24T07:47:51.540Z,
    Latitude: 42.6675,
    Longitude: 13.3238,
    Depth_Km: 6.5,
    Magnitude: 3.3 },
  { Time: 2016-08-24T07:52:25.710Z,
    Latitude: 42.7447,
    Longitude: 13.2827,
    Depth_Km: 7.9,
    Magnitude: 2.9 },
  { Time: 2016-08-24T07:52:43.210Z,
    Latitude: 42.6378,
    Longitude: 13.2313,
    Depth_Km: 10.9,
    Magnitude: 3.1 } ]
\end{minted}

\figpdf{figures/FIXME.png}{Seleting by Position}{f:dataforge-positional} % diagram showing head, tail, skip, and take

-   Usually more useful to filter to keep rows that pass some test
    -   Tests can be as complex as desired, but must work row by row

\begin{minted}{js}
const large = earthquakes.where(row => (row.Magnitude >= 5.0))
console.log(large.toArray())
\end{minted}

\begin{minted}{text}
[ { Time: 2016-08-24T07:36:32.000Z,
    Latitude: 42.6983,
    Longitude: 13.2335,
    Depth_Km: 8.1,
    Magnitude: 6 },
  { Time: 2016-08-24T08:33:28.890Z,
    Latitude: 42.7922,
    Longitude: 13.1507,
    Depth_Km: 8,
    Magnitude: 5.4 } ]
\end{minted}
    
\section{Aggregation}\label{s:dataforge-aggregate}

-   What are the average earthquake depth and magnitude?
    -   Produces an object rather than a dataframe

\begin{minted}{js}
const averageValues = earthquakes.summarize({
  Depth_Km: series => series.average(),
  Magnitude: series => series.average()
})
console.log(averageValues)
\end{minted}

\begin{minted}{text}
{ Depth_Km: 9.545614035087722, Magnitude: 2.5397243107769376 }
\end{minted}

-   Grouping is a little more complicated
-   First step: group by magnitude

\begin{minted}{js}
const groupedByMagnitude = earthquakes.groupBy(row => row.Magnitude)
console.log(`${groupedByMagnitude.count()} groups`)
console.log(groupedByMagnitude.head(2).toArray())
\end{minted}

\begin{minted}{text}
28 groups
[ DataFrame {
    configFn: null,
    content:
     { index: [ExtractElementIterable],
       values: [ExtractElementIterable],
       pairs: [Array],
       isBaked: false,
       columnNames: [ColumnNamesIterable] } },
  DataFrame {
    configFn: null,
    content:
     { index: [ExtractElementIterable],
       values: [ExtractElementIterable],
       pairs: [Array],
       isBaked: false,
       columnNames: [ColumnNamesIterable] } } ]
\end{minted}

-   Grouping returns an array of new dataframes, one per group
-   So let's calculate the average depth and magnitude for each

\begin{minted}{js}
const averaged = earthquakes
      .groupBy(row => row.Magnitude)
      .select(group => ({
        Magnitude: group.first().Magnitude,
        Depth_Km: group.deflate(row => row.Depth_Km).average()
      }))
      .inflate()
      .orderBy(row => row.Magnitude)
console.log(averaged.toArray())
\end{minted}

\begin{minted}{text}
[ { Magnitude: 2, Depth_Km: 9.901052631578946 },
  { Magnitude: 2.1, Depth_Km: 9.702083333333333 },
  { Magnitude: 2.2, Depth_Km: 9.843037974683545 },
  ...
  { Magnitude: 4.5, Depth_Km: 9.4 },
  { Magnitude: 5.4, Depth_Km: 8 },
  { Magnitude: 6, Depth_Km: 8.1 } ]
\end{minted}

-   That worked, but what are \texttt{deflate} and \texttt{inflate} and why are they needed?
-   \texttt{inflate} turns a series into a dataframe
-   \texttt{deflate} turns a column of a dataframe into a series

\figpdf{figures/FIXME.png}{Summarizing Groups}{f:dataforge-positional} % show steps in deflate and inflate

\section{In Real Life}\label{s:dataforge-real}

FIXME: extended example

\section{Exercises}\label{s:dataforge-exercises}

FIXME: read and write JSON instead of CSV

FIXME: sort descending

FIXME: how to do multi-column distinct?

FIXME: explain why we need \texttt{select} in the grouped aggregate example

\section*{Key Points}

\begin{itemize}
\item
  Create a \texttt{DataFrame} from an array of objects with identical keys,
  from a spec with \texttt{columnNames} and \texttt{rows} fields,
  or by parsing text that contains CSV or JSON.

\item
  If you're using a loop on a dataframe,
  you're doing the wrong thing.

\item
  Use method chaining to create pipelines
  that filter data and create new values from old.

\item
  Use grouping and aggregation to summarize data.
\end{itemize}

