\chapter{Using Data-Forge}\label{s:dataforge}

We have now seen how to do everything a data scientist would want to do in JavaScript
except actual data science.
This is unfortunately one of the areas where the language still lags behind R and Python,
but statistical libraries are now appearing,
and if the last twenty-five years have taught us anything,
it's not to underestimate JavaScript.

In this chapter we will look at a library called Data-Forge\index{Data-Forge}
that is designed for working with tabular data.
Data-Forge was inspired by Python's Pandas library,\index{Pandas}
but should be familiar to anyone who has worked with the tidyverse in R as well.\index{tidyverse}
Its \grefdex{g:dataframe}{\texttt{DataFrame}}{dataframe} class represents a table made up of named columns
and any number of rows.
Dataframes are \gref{g:immutable}{immutable}:
once a dataframe has been constructed,
its contents cannot be changed.
Instead,
every operation produces a new dataframe.
(Some clever behind-the-scenes data recycling makes this much more efficient than it sounds.)

Like Pandas and the tidyverse,
Data-Forge is designed to work on \gref{g:tidy-data}{tidy data}.
As defined in \cite{Wick2014}, tabular data is tidy if:

\begin{itemize}
\item
  Each column contains one statistical variable
  (i.e., one property that was measured or observed).

\item
  Each different observation is in a different row.

\item
  There is one table for each set of observations.

\item
  If there are multiple tables,
  each table has a column containing a unique key
  so that related data can be linked.
\end{itemize}

For example,
this data is not tidy:

\begin{longtable}{|l|l|l|l|l|}
  \hline
  \multicolumn{5}{|c|}{Rodent Pleurisy Rates} \\
  \hline
  & \multicolumn{2}{c|}{Female} & \multicolumn{2}{c|}{Male} \\
  \hline
  & 2018 & 2019 & 2018 & 2019 \\
  \hline
  Jan & 0.05 & 0.07 & 0.03 & 0.06 \\
  \hline
  Feb & 0.05 & 0.08 & 0.04 & 0.07 \\
  \hline
  Mar & 0.05 & 0.11 & 0.04 & 0.10 \\
  \hline
\end{longtable}

\noindent
but this data is:

\begin{longtable}{|l|l|l|l|}
  \hline
  Year & Month & Sex & Rate \\
  \hline
  2018 & Jan & Female & 0.05 \\
  \hline
  2018 & Feb & Female & 0.05 \\
  \hline
  2018 & Mar & Female & 0.05 \\
  \hline
  2018 & Jan & Male & 0.03 \\
  \hline
  2018 & Feb & Male & 0.04 \\
  \hline
  2018 & Mar & Male & 0.04 \\
  \hline
  2019 & Jan & Female & 0.07 \\
  \hline
  2019 & Feb & Female & 0.08 \\
  \hline
  2019 & Mar & Female & 0.11 \\
  \hline
  2019 & Jan & Male & 0.06 \\
  \hline
  2019 & Feb & Male & 0.07 \\
  \hline
  2019 & Mar & Male & 0.10 \\
  \hline
\end{longtable}

\section{Basic Operations}\label{s:dataforge-basics}

To get started,
we install Data-Forge using \texttt{npm install data-forge}
and then load the library and create a dataframe from a list of objects.
Each of these objects must use the same keys,
which become the names of the dataframe's columns:

\begin{minted}{js}
const DF = require('data-forge')

const fromObjects = new DF.DataFrame([
    {ones: 1, tens: 10},
    {ones: 2, tens: 20},
    {ones: 3, tens: 30}
])
console.log('fromObjects:\n', fromObjects)
\end{minted}

When we print the dataframe,
we see this rather complex structure:

\begin{minted}{text}
fromObjects:
 DataFrame {
  configFn: null,
  content:
   { index: CountIterable {},
     values: [ [Object], [Object], [Object] ],
     pairs: MultiIterable { iterables: [Array] },
     isBaked: true,
     columnNames: [ 'ones', 'tens' ] } }
\end{minted}

Each column is stored as a \texttt{Series} object;
once in a while,
we will need to work with these objects directly
instead of with the dataframe as a whole.
(\figref{f:dataforge-rows-and-columns}).
If we want to see the actual data,
we need to convert the dataframe back to an array of objects:

\figpdf{figures/dataforge-rows-and-columns.pdf}{How Tables Are Stored}{f:dataforge-rows-and-columns}

\begin{minted}{js}
console.log('fromObjects as array:\n', fromObjects.toArray())
\end{minted}

\begin{minted}{text}
fromObjects as array:
 [ { ones: 1, tens: 10 },
  { ones: 2, tens: 20 },
  { ones: 3, tens: 30 } ]
\end{minted}

We can instead create a dataframe by providing the names of the columns in one list
and the rows' values in another:

\begin{minted}{js}
const fromSpec = new DF.DataFrame({
  columnNames: ['ones', 'tens'],
  rows: [
    [4, 40],
    [5, 50],
    [6, 60]
  ]
})
console.log('fromSpec as array:\n', fromSpec.toArray())
\end{minted}

\begin{minted}{text}
fromSpec as array:
 [ { ones: 4, tens: 40 },
  { ones: 5, tens: 50 },
  { ones: 6, tens: 60 } ]
\end{minted}

However,
we usually won't create a dataframe directly like this;
instead,
we will read data from a file or a database.
Data-Forge provides a function called \texttt{fromCSV} for doing this:

\begin{minted}{js}
const text = `ones,tens
7,70
8,80
9,90`
const fromText = DF.fromCSV(text)
console.log('fromText as array:\n', fromText.toArray())
\end{minted}

\begin{minted}{text}
fromText as array:
 [ { ones: '7', tens: '70' },
  { ones: '8', tens: '80' },
  { ones: '9', tens: '90' } ]
\end{minted}

However we create our dataframe,
we can ask it for its columns' names:

\begin{minted}{js}
console.log(data.getColumnNames())
\end{minted}

\begin{minted}{text}
[ 'ones', 'tens' ]
\end{minted}

\noindent
or get its content as a list of lists (rather than as a list of objects):

\begin{minted}{js}
console.log(data.toRows())
\end{minted}

\begin{minted}{text}
[ [ 1, 10 ], [ 2, 20 ], [ 3, 30 ] ]
\end{minted}

We can also process the rows using a \texttt{for} loop:

\begin{minted}{js}
for (let row of data) {
  console.log(row)
}
\end{minted}

\begin{minted}{text}
{ ones: 1, tens: 10 }
{ ones: 2, tens: 20 }
{ ones: 3, tens: 30 }
\end{minted}

\noindent
or its \texttt{forEach} method:

\begin{minted}{js}
data.forEach(row => {
  console.log(row)
})
\end{minted}

\begin{minted}{text}
{ ones: 1, tens: 10 }
{ ones: 2, tens: 20 }
{ ones: 3, tens: 30 }
\end{minted}

\noindent
However,
a good rule of thumb is that if you're using a loop on a dataframe,
you're doing the wrong thing:
you should instead use the methods described below.

\section{Doing Calculations}\label{s:dataforge-calc}

Suppose we want to add a new column to a dataframe---or rather,
create a new dataframe with an extra column,
since we can't modify a dataframe in place.
To do this,
we create a new \texttt{Series} object to represent that column,
then use \texttt{withSeries} to construct our result:

\begin{minted}{js}
const double_oh = new DF.Series([100, 200, 300])

const withHundreds = data.withSeries({hundreds: double_oh})
console.log(withHundreds.toArray())
\end{minted}

\begin{minted}{text}
[ { ones: 1, tens: 10, hundreds: 100 },
  { ones: 2, tens: 20, hundreds: 200 },
  { ones: 3, tens: 30, hundreds: 300 } ]
\end{minted}

Just as we usually create dataframes by reading data from external sources,
we will usually create new columns from existing values.
As you probably won't be surprised to learn,
we tell Data-Forge how to do this by writing callback functions.
Since we often want to create several new columns at once,
we give the \texttt{generateSeries} method an object
whose keys are the names of the new columns
and whose values are callbacks taking a row as input and producing a new value as output:

\begin{minted}{js}
const sumsAndProducts = data.generateSeries({
  sum: row => row.ones + row.tens,
  product: row => row.ones * row.tens
})
console.log(sumsAndProducts.toArray())
\end{minted}

\begin{minted}{text}
[ { ones: 1, tens: 10, sum: 11, product: 10 },
  { ones: 2, tens: 20, sum: 22, product: 40 },
  { ones: 3, tens: 30, sum: 33, product: 90 } ]
\end{minted}

We can also get rid of columns entirely using \texttt{dropSeries}:

\begin{minted}{js}
const justResults = sumsAndProducts.dropSeries(["ones", "tens"])
console.log(justResults.toArray())
\end{minted}

\begin{minted}{text}
[ { sum: 11, product: 10 },
  { sum: 22, product: 40 },
  { sum: 33, product: 90 } ]
\end{minted}

Since every dataframe method returns a dataframe,
we can use \gref{g:method-chaining}{method chaining} to combine operations (\figref{f:dataforge-method-chaining}).
We have seen this technique before with chains of \texttt{.then} calls on promises;
here,
it is used like pipes in the Unix command line
or the pipe operator \texttt{{\%}{\textgreater}{\%}} in modern R code:

\begin{minted}{js}
const result = data
  .withSeries({hundreds: double_oh})
  .generateSeries({
    sum: row => row.ones + row.tens + row.hundreds
  })
  .dropSeries(["ones", "tens", "hundreds"])
  .toArray()
\end{minted}

\figpdf{figures/dataforge-method-chaining.pdf}{A More Complicated Pipeline}{f:dataforge-method-chaining}

To make results easier to understand,
we will often want to sort our data.
Suppose we have a file containing the red-green-blue values for several colors:

\begin{minted}{text}
name,red,green,blue
maroon,128,0,0
lime,0,255,0
navy,0,0,128
yellow,255,255,0
fuchsia,255,0,255
aqua,0,255,255
\end{minted}

\noindent
We can pass the name of this file to our program as a command-line argument,
read it (remembering to set the encoding to UTF-8 so that we get characters rather than raw bytes),
and then display it:

\begin{minted}{js}
const fs = require('fs')
const DF = require('data-forge')

const text = fs.readFileSync(process.argv[2], 'utf-8')
const colors = DF.fromCSV(text)
console.log(colors.toArray())
\end{minted}

\begin{minted}{text}
[ { name: 'maroon', red: '128', green: '0', blue: '0' },
  { name: 'lime', red: '0', green: '255', blue: '0' },
  { name: 'navy', red: '0', green: '0', blue: '128' },
  { name: 'yellow', red: '255', green: '255', blue: '0' },
  { name: 'fuchsia', red: '255', green: '0', blue: '255' },
  { name: 'aqua', red: '0', green: '255', blue: '255' } ]
\end{minted}

If we want to see the colors in alphabetical order,
we call \texttt{orderBy} with a callback that gives Data-Forge the value to sort by:

\begin{minted}{js}
const sorted = colors.orderBy(row => row.name)
console.log(sorted.toArray())
\end{minted}

\begin{minted}{text}
[ { name: 'aqua', red: '0', green: '255', blue: '255' },
  { name: 'fuchsia', red: '255', green: '0', blue: '255' },
  { name: 'lime', red: '0', green: '255', blue: '0' },
  { name: 'maroon', red: '128', green: '0', blue: '0' },
  { name: 'navy', red: '0', green: '0', blue: '128' },
  { name: 'yellow', red: '255', green: '255', blue: '0' } ]
\end{minted}

\noindent
To sub-sort (\figref{f:dataforge-sorting}) by another column we use \texttt{thenBy}:

\begin{minted}{js}
const doubleSorted = colors
      .orderBy(row => row.green)
      .thenBy(row => row.blue)
      .dropSeries(['name', 'red'])
console.log(doubleSorted.toArray())
\end{minted}

\begin{minted}{text}
[ { green: '0', blue: '0' },
  { green: '0', blue: '128' },
  { green: '0', blue: '255' },
  { green: '255', blue: '0' },
  { green: '255', blue: '0' },
  { green: '255', blue: '255' } ]
\end{minted}

\figpdf{figures/dataforge-sorting.pdf}{Sorting and Sub-sorting}{f:dataforge-sorting}

We can remove duplicates with \texttt{distinct}:

\begin{minted}{js}
const notTheSame = colors.distinct(row => row.red)
console.log(notTheSame.toArray())
\end{minted}

\begin{minted}{text}
[ { name: 'maroon', red: '128', green: '0', blue: '0' },
  { name: 'lime', red: '0', green: '255', blue: '0' },
  { name: 'yellow', red: '255', green: '255', blue: '0' } ]
\end{minted}

\noindent
but this is trickier than it appears.
Each row does indeed have a distinct \texttt{red} value,
but Data-Forge gets to decide which row to keep from each subset.
What's more surprising is that multi-column \texttt{distinct} \emph{doesn't} work:

\begin{minted}{js}
const multiColumn = colors
      .distinct(row => [row.red, row.green])
      .orderBy(row => row.red)
      .thenBy(row => row.green)
console.log(multiColumn.toArray())
\end{minted}

\begin{minted}{text}
[ { name: 'navy', red: '0', green: '0', blue: '128' },
  { name: 'lime', red: '0', green: '255', blue: '0' },
  { name: 'aqua', red: '0', green: '255', blue: '255' },
  { name: 'maroon', red: '128', green: '0', blue: '0' },
  { name: 'fuchsia', red: '255', green: '0', blue: '255' },
  { name: 'yellow', red: '255', green: '255', blue: '0' } ]
\end{minted}

This isn't Data-Forge's fault.
In JavaScript,
two arrays are only equal if they're the same object,
i.e., \texttt{[0] === [0]} is \texttt{false}.
We will explore ways of doing multi-column \texttt{distinct} in the exercises.

\section{Subsets}\label{s:dataforge-subset}

You may have noticed that the color values in the table above are strings rather than numbers.
If we want Data-Forge to convert values to more useful types,\index{dataframe!type conversion}
we can use the methods \texttt{parseDates}, \texttt{parseFloats}, and so on.
The program below does this for a subset of
\hreffoot{https://earthquake.usgs.gov/earthquakes/feed/v1.0/csv.php}{USGS data about earthquakes in August 2016}:

\begin{minted}{js}
const fs = require('fs')
const DF = require('data-forge')

const text = fs.readFileSync('earthquakes.csv', 'utf-8')
const earthquakes = DF
      .fromCSV(text)
      .parseDates('Time')
      .parseFloats(['Latitude', 'Longitude', 'Depth_Km', 'Magnitude'])
console.log('Data has', earthquakes.count(), 'rows')
\end{minted}

\begin{minted}{text}
Data has 798 rows
\end{minted}

Whether we convert it or not,
we will often want to work with subsets of data.
We can select values by position using \texttt{head} and \texttt{tail}\index{dataframe!select by location}
(which are named after classic Unix commands):

\begin{minted}{js}
console.log(earthquakes.head(3).toArray())
\end{minted}

\begin{minted}{text}
[ { Time: 2016-08-24T07:36:32.000Z,
    Latitude: 42.6983,
    Longitude: 13.2335,
    Depth_Km: 8.1,
    Magnitude: 6 },
  { Time: 2016-08-24T07:37:26.580Z,
    Latitude: 42.7123,
    Longitude: 13.2533,
    Depth_Km: 9,
    Magnitude: 4.5 },
  { Time: 2016-08-24T07:40:46.590Z,
    Latitude: 42.7647,
    Longitude: 13.1723,
    Depth_Km: 9.7,
    Magnitude: 3.8 } ]
\end{minted}

\begin{minted}{js}
console.log(earthquakes.tail(3).toArray())
\end{minted}

\begin{minted}{text}
[ { Time: 2016-08-26T10:09:45.380Z,
    Latitude: 42.6953,
    Longitude: 13.2363,
    Depth_Km: 9.5,
    Magnitude: 2.3 },
  { Time: 2016-08-26T10:11:55.960Z,
    Latitude: 42.6163,
    Longitude: 13.2985,
    Depth_Km: 11,
    Magnitude: 2.1 },
  { Time: 2016-08-26T10:21:09.870Z,
    Latitude: 42.6153,
    Longitude: 13.2952,
    Depth_Km: 7.5,
    Magnitude: 3 } ]
\end{minted}

If we want data from the middle of the table,
we can skip a few rows and then take as many as we want (\figref{f:dataforge-positional}):

\begin{minted}{js}
console.log(earthquakes.skip(10).take(3).toArray())
\end{minted}

\begin{minted}{text}
[ { Time: 2016-08-24T07:47:51.540Z,
    Latitude: 42.6675,
    Longitude: 13.3238,
    Depth_Km: 6.5,
    Magnitude: 3.3 },
  { Time: 2016-08-24T07:52:25.710Z,
    Latitude: 42.7447,
    Longitude: 13.2827,
    Depth_Km: 7.9,
    Magnitude: 2.9 },
  { Time: 2016-08-24T07:52:43.210Z,
    Latitude: 42.6378,
    Longitude: 13.2313,
    Depth_Km: 10.9,
    Magnitude: 3.1 } ]
\end{minted}

\figpdf{figures/dataforge-positional.pdf}{Selecting by Position}{f:dataforge-positional}

However,
it's far more common to select rows by the values they contain rather than by their position.\index{dataframe!select by value}
Just like the \texttt{Array.filter} method we met way back in \secref{s:callbacks-functional},
we do this by giving Data-Forge a callback function that tells it whether a given row should be kept or not.
This text can be as complex as desired,
but must work row by row:
we cannot make a decision about one row based on the values in the rows before it or after it.

\begin{minted}{js}
const large = earthquakes.where(row => (row.Magnitude >= 5.0))
console.log(large.toArray())
\end{minted}

\begin{minted}{text}
[ { Time: 2016-08-24T07:36:32.000Z,
    Latitude: 42.6983,
    Longitude: 13.2335,
    Depth_Km: 8.1,
    Magnitude: 6 },
  { Time: 2016-08-24T08:33:28.890Z,
    Latitude: 42.7922,
    Longitude: 13.1507,
    Depth_Km: 8,
    Magnitude: 5.4 } ]
\end{minted}
    
\section{Aggregation}\label{s:dataforge-aggregate}

Working with individual observations is all very well,
but if we want to understand our data,
we need to look at its aggregate properties.\index{dataframe!aggregation}
If,
for example,
we want to know the average depth and magnitude of our earthquakes,
we use the \texttt{summarize} method:

\begin{minted}{js}
const averageValues = earthquakes.summarize({
  Depth_Km: series => series.average(),
  Magnitude: series => series.average()
})
console.log(averageValues)
\end{minted}

\begin{minted}{text}
{ Depth_Km: 9.545614035087722, Magnitude: 2.5397243107769376 }
\end{minted}

\noindent
As with \texttt{filter},
we can do many calculations at once by giving \texttt{summarize} several callbacks.
The keys of the object we pass in specify the columns we want to aggregate;
the callbacks invoke methods of the \texttt{Series} class that Data-Forge uses to store individual columns.
Instead of producing a dataframe with a single row,
\texttt{summarize} produces an object whose keys match the names of the columns in our original dataframe.

Aggregation is often combined with grouping:
for example,
we may want to calculate the average weight of rats of different breeds
or the distribution of votes by province.
The first step is to group the data:

\begin{minted}{js}
const groupedByMagnitude = earthquakes.groupBy(row => row.Magnitude)
console.log(`${groupedByMagnitude.count()} groups`)
console.log(groupedByMagnitude.head(2).toArray())
\end{minted}

\begin{minted}{text}
28 groups
[ DataFrame {
    configFn: null,
    content:
     { index: [ExtractElementIterable],
       values: [ExtractElementIterable],
       pairs: [Array],
       isBaked: false,
       columnNames: [ColumnNamesIterable] } },
  DataFrame {
    configFn: null,
    content:
     { index: [ExtractElementIterable],
       values: [ExtractElementIterable],
       pairs: [Array],
       isBaked: false,
       columnNames: [ColumnNamesIterable] } } ]
\end{minted}

As the output shows,
\texttt{groupBy} returns an array containing one new dataframe for each group in our original data.
Here's how we find the average depth of earthquakes according to magnitude:

\begin{minted}{js}
const averaged = earthquakes
      .groupBy(row => row.Magnitude)
      .select(group => ({
        Magnitude: group.first().Magnitude,
        Depth_Km: group.deflate(row => row.Depth_Km).average()
      }))
      .inflate()
      .orderBy(row => row.Magnitude)
console.log(averaged.toArray())
\end{minted}

\begin{minted}{text}
[ { Magnitude: 2, Depth_Km: 9.901052631578946 },
  { Magnitude: 2.1, Depth_Km: 9.702083333333333 },
  { Magnitude: 2.2, Depth_Km: 9.843037974683545 },
  ...
  { Magnitude: 4.5, Depth_Km: 9.4 },
  { Magnitude: 5.4, Depth_Km: 8 },
  { Magnitude: 6, Depth_Km: 8.1 } ]
\end{minted}

Going through this step by step:

\begin{enumerate}

\item
  The \texttt{groupBy} call produces a list of 28 dataframes,
  one for each distinct value of \texttt{Magnitude}.

\item
  \texttt{select} then converts each of these dataframes into an object
  whose \texttt{Magnitude} is equal to the magnitude of the group's first element
  and whose \texttt{Depth\_Km} is the average of the depths.
  \begin{itemize}
  \item
    We can use the magnitude of the group's first element because all of the magnitudes in the group are the same.
  \item
    We must also remember to use \texttt{deflate} to turn a column of a dataframe into a \texttt{Series}
    so that we can then call \texttt{average}.
  \end{itemize}

\item
  The output of \texttt{select} is a \texttt{Series} of two-valued objects,
  so we must call \texttt{inflate} to convert it back to a \texttt{DataFrame}.

\item
  Finally,
  we order by the magnitude of the earthquakes to produce our output.
  
\end{enumerate}

\figref{f:dataforge-summarizing} shows these steps graphically.
It's easy to forget the \texttt{inflate} and \texttt{deflate} steps at first
(we did when writing this example),
but they quickly become habitual.

\figpdf{figures/dataforge-summarizing.pdf}{Summarizing Groups}{f:dataforge-summarizing}

\section{In Real Life}\label{s:dataforge-real}

To wrap up our exploration of Data-Forge,
we will explore data from \url{http://dataisplural/data.world}
to find the annual recreational visits to national parks in the United States for the last ten years.
Our strategy is to:

\begin{enumerate}
\item
  import the data,
\item
  inspect the columns to make sure the data is clean,
\item
  fix any problems we notice,
\item
  group the data by year and summarize the total visitors, and then
\item
  filter to keep only the years that are greater than 2009.
\end{enumerate}

We'll start by reading the data and looking at the first couple of rows:

\begin{minted}{js}
const fs = require('fs')
const DF = require('data-forge')

const text = fs.readFileSync('../../data/national_parks.csv', 'utf-8')
const raw = DF.fromCSV(text)
console.log(raw.head(2).toArray())
\end{minted}

\begin{minted}{text}
[ { year: '1904',
    gnis_id: '1163670',
    geometry: 'POLYGON',
    metadata: 'NA',
    number_of_records: '1',
    parkname: 'Crater Lake',
    region: 'PW',
    state: 'OR',
    unit_code: 'CRLA',
    unit_name: 'Crater Lake National Park',
    unit_type: 'National Park',
    visitors: '1500' },
  { year: '1941',
    gnis_id: '1531834',
    geometry: 'MULTIPOLYGON',
    metadata: 'NA',
    number_of_records: '1',
    parkname: 'Lake Roosevelt',
    region: 'PW',
    state: 'WA',
    unit_code: 'LARO',
    unit_name: 'Lake Roosevelt National Recreation Area',
    unit_type: 'National Recreation Area',
    visitors: '0' } ]
\end{minted}

It's always good to look at the structure of the data
before we dive into any analysis.
The dataframe method \texttt{detectTypes} shows us the frequency of data types in our dataframe:

\begin{minted}{js}
const typesDf = raw.detectTypes() 
console.log(typesDf.toString())
\end{minted}

\begin{minted}{text}
__index__  Type    Frequency  Column
---------  ------  ---------  -----------------
0          string  100        year
1          string  100        gnis_id
2          string  100        geometry
3          string  100        metadata
4          string  100        number_of_records
5          string  100        parkname
6          string  100        region
7          string  100        state
8          string  100        unit_code
9          string  100        unit_name
10         string  100        unit_type
11         string  100        visitors
\end{minted}

We want numbers instead of strings for \texttt{year} and \texttt{visitors},
but before we transform them,
let's check and see if any values in those columns are \texttt{NA},
meaning ``not available'':

\begin{minted}{js}
const cleaned = raw
      .where(row => ((row.year != 'NA') && (row.visitors != 'NA')))
console.log(`${raw.count()} original rows and ${cleaned.count()} cleaned rows`)
\end{minted}

\begin{minted}{text}
21560 original rows and 21556 cleaned rows
\end{minted}

Four rows contain missing values.
We probably wouldn't spot them if we scrolled through the data ourselves,
so it's good that we let the computer do the work.
Let's remove those four rows and convert the two columns of interest from strings to numbers:

\begin{minted}{js}
const data = raw
      .where(row => ((row.year != 'NA') && (row.visitors != 'NA')))
      .parseFloats(['year', 'visitors'])
console.log(`${data.count()} rows`)
console.log(data.detectTypes().toString())
\end{minted}

\begin{minted}{text}
21556 rows
__index__  Type    Frequency  Column
---------  ------  ---------  -----------------
0          number  100        year
...        ...     ...        ...
11         number  100        visitors
\end{minted}

This looks good:
we have dropped the four offending rows and everything else is a number.
With clean data,
we can now group by year
and find the total number of visitors in each year:

\begin{minted}{js}
const annual = data
      .groupBy(row => row.year)
      .select(group => ({
        year: group.first().year,
        visitors: group.deflate(row => row.visitors).sum()
      }))
      .inflate()
      .orderBy(row => row.year)
console.log(annual.toString())
\end{minted}

\begin{minted}{text}
__index__  year  visitors   
---------  ----  -----------
0          1904  120690     
112        1905  140954     
88         NaN   13764633135
55         1906  30569      
113        1907  32935      
61         1908  42768      
111        1909  60899      
110        1910  173416     
...        ...   ...
23         2011  276799292  
91         2012  281392715  
70         2013  271305455  
89         2014  290105230  
78         2015  304730566  
75         2016  328483428  
\end{minted}

Uh oh.
What's that \texttt{NaN} doing there in the third row?
Are there still some missing values in the \texttt{year} column?

\begin{minted}{js}
const numNan = data
      .where(row => (isNaN(row.year) || isNaN(row.visitors)))
      .count()
console.log(`${numNan} rows have NaN`)
\end{minted}

\begin{minted}{text}
382 rows have NaN
\end{minted}

The exercises will look at where these non-numbers have come from and how we should handle them.
Even if you don't live close to a national park,
we encourage you to take a break and step outside before tackling the exercises below.

\section{Exercises}\label{s:dataforge-exercises}

\exercise{Other Data Formats}

Write a short program that reads data from \texttt{colors.csv},
converts the red-green-blue values to numbers,
and saves the result as JSON.
Once that is working,
write another program that reads the JSON file and converts it back to CSV.
Are there any differences between your second program's output
and your original CSV file?

\exercise{Directional Sorting}

Modify the program in \secref{s:dataforge-calc} to sort color values
by \emph{increasing} red and \emph{decreasing} green.

\exercise{Multi-Column Distinct}

\secref{s:dataforge-calc} pointed out that we cannot use \texttt{distinct}
to find rows with distinct combinations of values.
What \emph{can} we do?
To show that your idea works,
write a program that gets distinct combinations of red and green values from the color data:

\begin{minted}{text}
red,green
0,0
0,255
128,0
255,0
255,255
\end{minted}

\exercise{Revisiting Data Manipulation}

Back in the chapter on data manipulation,
we aggregated \texttt{surveys.csv} to find the minimum, maximum, and average values 
for \texttt{year}, \texttt{hindfoot\_length}, and \texttt{weight}.
Repeat this exercise using the methods of \texttt{data-forge}.

\exercise{Grouping and Aggregating}

Retrace the steps in the example that calculated
the average depth of earthquakes of different magnitudes
and show the data structure after each method call.
Are they all dataframes,
or are other data structures created or manipulated as well?

\exercise{Not a Number}

Write a short pipeline that prints out the values in the \texttt{year} and \texttt{visitors} columns
that wind up being converted to \texttt{NaN}.
(Hint: copy the columns, convert the copies to numbers, filter to keep those rows, then print the original values.)

\section*{Key Points}

\begin{itemize}
\item
  Create a \texttt{DataFrame} from an array of objects with identical keys,
  from a spec with \texttt{columnNames} and \texttt{rows} fields,
  or by parsing text that contains CSV or JSON.

\item
  If you're using a loop on a dataframe,
  you're doing the wrong thing.

\item
  Use method chaining to create pipelines
  that filter data and create new values from old.

\item
  Use grouping and aggregation to summarize data.
\end{itemize}

