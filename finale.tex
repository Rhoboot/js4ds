\chapter{Finale}\label{s:finale}

We have come a long way since \texttt{console.log('hello,\ world')} in \chapref{s:basics}.
Callbacks and promises,
JSON and web servers,
packaging, unit tests, and visualization:
every modern language can do them,
but JavaScript is an increasingly popular choice.
Yes,
it has its flaws,
but if we avoid some of the legacy features in \appref{s:legacy}
it's both usable and powerful.

Our journey doesn't stop here, though.
The appendices explore some next steps,
such as logging what our server does (\appref{s:logging})
and using a relational database (\appref{s:db}) instead of a text file
as a data store.
Beyond that,
you could look at more advanced techniques in JavaScript \cite{Have2018},
explore the full power of the \hreffoot{https://d3js.org/}{D3} library for interactive visualization \cite{Meek2017},
dive into data wrangling \cite{Davi2018},
or start over completely the way JavaScript programmers do every eight months
and rewrite everything with \hreffoot{https://developer.mozilla.org/en-US/docs/Web/Web\_Components}{Web Components}.
Whatever you do,
we hope that this tutorial has helped you get started.

Contributions of all kinds are welcome,
from errata and minor improvements to entirely new sections and chapters.
Please \hreffoot{https://github.com/software-tools-in-javascript/js4ds/issues}{file an issue}
or \hreffoot{https://github.com/software-tools-in-javascript/js4ds/pulls}{submit a pull request}
in \hreffoot{https://github.com/software-tools-in-javascript/js4ds/}{our GitHub repository}.
Everyone whose work is incorporated will be acknowledged.
Please see the contributors' guide for more information,
and please note that all contributors are required to abide by
our Code of Conduct.

\section*{Key Points}

\begin{itemize}
\item
  We have learned a lot.
\item
  Contributions are very welcome.
\end{itemize}

