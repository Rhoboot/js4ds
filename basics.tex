\chapter{Basic Features}\label{s:basics}

This lesson introduces the core features of JavaScript,
including how to run programs,
the language's basic data types,
arrays and objects,
loops,
conditionals,
functions,
and modules.
All of these concepts should be familiar if you have programmed before.

\section{Hello, World}\label{s:basics-hello-world}

Use your favorite text editor to put the following line in a file called \texttt{hello.js}:

\begin{minted}{js}
console.log('hello, world')
\end{minted}

\texttt{console} is a built-in \gref{g:module}{module} that provides basic printing services
(among other things).
As in many languages,
we use the \gref{g:dotted-notation}{dotted notation} \texttt{X.Y} to get part \texttt{Y} of thing \texttt{X}---in this case,
to get \texttt{console}'s \texttt{log} function.
\gref{g:string}{Character strings} like \texttt{'hello,\ world'} can be written with either single quotes or double quotes,
so long as the quotation marks match,
and semi-colons at the ends of statements are now (mostly) optional.

To run a program,
type \texttt{node\ program\_name.js} at the command line.
(We will preface shell commands with \texttt{\$} to make them easier to spot.)

\begin{minted}{shell}
$ node src/basics/hello.js
\end{minted}

\begin{minted}{text}
hello, world
\end{minted}

\section{Basic Data Types}\label{s:basics-data-types}

JavaScript has the usual datatypes,
though unlike C, Python, and many other languages,
there is no separate type for integers:
it stores all numbers as 64-bit floating-point values,
which is accurate up to about 15 decimal digits.
We can check this using \texttt{typeof},
which returns a string.
(Note: \texttt{typeof} is an operator,
\emph{not} a function:
we apply it to something by typing a space followed by
the name of the thing we'd like to check the type of,
e.g. \texttt{typeof dress}, as opposed to \texttt{typeof(dress)}.)
We use it alongside \texttt{const} below,
which itself is helpful when we want to give a name to a
\gref{g:constant}{constant} value:

\begin{minted}{js}
const aNumber = 123.45
console.log('the type of', aNumber, 'is', typeof aNumber)
\end{minted}

\begin{minted}{text}
the type of 123.45 is number
\end{minted}

\begin{minted}{js}
const anInteger = 123
console.log('the type of', anInteger, 'is', typeof anInteger)
\end{minted}

\begin{minted}{text}
the type of 123 is number
\end{minted}

We have already met strings,
which may contain any \gref{g:unicode}{Unicode} character:

\begin{minted}{js}
const aString = 'some text'
console.log('the type of', aString, 'is', typeof aString)
\end{minted}

\begin{minted}{text}
the type of some text is string
\end{minted}

Functions are also a type of data,
a fact whose implications we will explore in \chapref{s:callbacks}:

\begin{minted}{js}
console.log('the type of', console.log, 'is', typeof console.log)
\end{minted}

\begin{minted}{text}
the type of function () { [native code] } is function
\end{minted}

Rather than showing the other basic types one by one,
we will put three values in a list and loop over it:

\begin{minted}{js}
const otherValues = [true, undefined, null]
for (let value of otherValues) {
  console.log('the type of', value, 'is', typeof value)
}
\end{minted}

\begin{minted}{text}
the type of true is boolean
the type of undefined is undefined
the type of null is object
\end{minted}

\noindent
As the example above shows, we create an array of values to loop through called \texttt{otherValues}.
We initiate our loop with the word \texttt{for}.
Within the parentheses, \texttt{let} creates a variable called \texttt{value} to iterate over each element within \texttt{otherValues},
and \texttt{value} is the changing array value of \texttt{otherValues}.
Finally, within the curly braces we perform our desired operation on every value.

Note that we use \texttt{let} rather than the older \texttt{var} and \texttt{of} and not \texttt{in}:
the latter returns the indexes of the collection (e.g., 0, 1, 2),
which has some traps for the unwary (\appref{s:legacy-iteration}).
Note also that indexing starts from 0 rather than 1,
and that indentation is optional and for readability purposes only.
This may be different from the language that you're used to.

\begin{aside}{Constants versus Variables}
  You should make things constants unless they really need to be variables
  because it's easier for both people and computers to keep track of things
  that are defined once and never change.
\end{aside}

After all this,
the types themselves are somewhat anticlimactic.
JavaScript's \texttt{boolean} type can be either \texttt{true} or \texttt{false},
though we will see below that other things can be treated as Booleans.
\texttt{undefined} means ``hasn't been given a value'',
while \texttt{null} means ``has a value, which is nothing''.

\section{Control Flow}\label{s:basics-control-flow}

We have already seen \texttt{for} loops and flat arrays,
so let's have a look at nested arrays and conditionals.
We start with arrays that contain other arrays,
which are usually processed by \gref{g:nested-loop}{nested loops}:

\begin{minted}{js}
const nested = [['northwest', 'northeast'],
                ['southwest', 'southeast']]
for (let outer of nested) {
  for (let inner of outer) {
    console.log(inner)
  }
}
\end{minted}

\begin{minted}{text}
northwest
northeast
southwest
southeast
\end{minted}

The \gref{g:inner-loop}{inner loop} runs a complete cycle of iterations
for each iteration of the \gref{g:outer-loop}{outer loop}.
Each value assigned to the variable \texttt{outer} is a pair,
so each value assigned to \texttt{inner} is one of the two strings from that pair
(\figref{f:basics-traversal}).

\figpdf{figures/basics-traversal.pdf}{Nested Loop Traversal}{f:basics-traversal}

A JavaScript program can also make choices:
it executes the \gref{g:body-statement}{body} of an \texttt{if} statement
if and only if the \gref{g:condition}{condition} is true.
Each \texttt{if} can have an \texttt{else},
whose body is only executed if the condition \emph{isn't} true:

\begin{minted}{js}
const values = [0, 1, '', 'text', undefined, null, [], [2, 3]]
for (let element of values) {
  if (element) {
    console.log(element, 'of type', typeof element, 'is truthy')
  } else {
    console.log(element, 'of type', typeof element, 'is falsy')
  }
}
\end{minted}

\begin{minted}{text}
0 of type number is falsy
1 of type number is truthy
 of type string is falsy
text of type string is truthy
undefined of type undefined is falsy
null of type object is falsy
 of type object is truthy
2,3 of type object is truthy
\end{minted}

This example shows that arrays are \gref{g:heterogeneous}{heterogeneous},
i.e.,
that they can contain values of many different types.
It also shows that JavaScript has some rather odd ideas about the nature of truth.
0 is \gref{g:falsy}{falsy}, while all other numbers are \gref{g:truthy}{truthy};
similarly,
the empty string is falsy and all other strings are truthy.
\texttt{undefined} and \texttt{null} are both falsy,
as most programmers would expect.

But as the last two lines of output show,
an empty array is truthy,
which is different from its treatment in most programming languages.
The argument made by JavaScript's advocates is that an empty array is there,
it just happens to be empty,
but this behavior is still a common cause of bugs.
When testing an array,
check that \texttt{Array.length} is zero.
(Note that this is a \gref{g:property}{property},
not a \gref{g:method}{method},
i.e.,
it should be treated as a variable,
not called like a function.)

\begin{aside}{Safety Tip}
  Always use \texttt{===} (triple equals) and \texttt{!==} when testing for equality and inequality in JavaScript.
  \texttt{==} and \texttt{!=} do type conversion,
  which can produce some ugly surprises (\secref{s:legacy-equality}).
\end{aside}

\section{Formatting Strings}\label{s:basics-formatting}

Rather than printing multiple strings and expressions,
we can \gref{g:string-interpolation}{interpolate} values into a back-quoted string.
(We have to use back quotes because this feature was added to JavaScript
long after the language was first created.)
As the example below shows,
the value to be interpolated is put in \texttt{\$\{...\}},
and can be any valid JavaScript expression,
including a function or method call.

\begin{minted}{js}
for (let color of ['red', 'green', 'blue']) {
  const message = `color is ${color}`
  console.log(message, `and capitalized is ${color.toUpperCase()}`)
}
\end{minted}

\begin{minted}{text}
color is red and capitalized is RED
color is green and capitalized is GREEN
color is blue and capitalized is BLUE
\end{minted}

\noindent
This allows us to succinctly add variables to a string instead of:

\begin{minted}{js}
const message = "color is" + color + "and capitalized is " + color.toUpperCase()
\end{minted}

\section{Objects}\label{s:basics-objects}

An \gref{g:object}{object} in JavaScript is a collection of key-value pairs,
and is equivalent in simple cases to what Python would call a dictionary.
It's common to visualize an object as a two-column table
with the keys in one column and the values in another (\figref{f:basics-object-table}).
The keys do not have to be strings,
but almost always are;
the values can be anything.
We can create an object by putting key-value pairs in curly brackets;
there must be a colon between the key and the value,
and pairs must be separated by commas like the elements of arrays:

\figpdf{figures/basics-object-table.pdf}{Objects in Memory}{f:basics-object-table}

\begin{minted}{js}
const creature = {
  'order': 'Primates',
  'family': 'Callitrichidae',
  'genus': 'Callithrix',
  'species': 'Jacchus'
}

console.log(`creature is ${creature}`)
console.log(`creature.genus is ${creature.genus}`)
for (let key in creature) {
  console.log(`creature[${key}] is ${creature[key]}`)
}
\end{minted}

\begin{minted}{text}
creature is [object Object]
creature.genus is Callithrix
creature[order] is Primates
creature[family] is Callitrichidae
creature[genus] is Callithrix
creature[species] is Jacchus
\end{minted}

The type of an object is always \texttt{object}.
We can get the value associated with a key using \texttt{object[key]},
but if the key has a simple name,
we can use \texttt{object.key} instead.
Note that the square bracket form can be used with variables for keys,
but the dotted notation cannot:
i.e.,
\texttt{creature.genus} is the same as \texttt{creature['genus']},
but the assignment \texttt{g\ =\ 'genus'} followed by \texttt{creature.g} does not work.

Because string keys are so common,
and because programmers use simple names so often,
JavaScript allows us to create objects without quoting the names of the keys:

\begin{minted}{js}
const creature = {
  order: 'Primates',
  family: 'Callitrichidae',
  genus: 'Callithrix',
  species: 'Jacchus'
}
\end{minted}

\texttt{[object\ Object]} is not particularly useful output when we want to see what an object contains.
To get a more helpful string representation,
use \texttt{JSON.stringify(object)}:

\begin{minted}{js}
console.log(JSON.stringify(creature))
\end{minted}

\begin{minted}{text}
{"order":"Primates","family":"Callitrichidae",
 "genus":"Callithrix","species":"Jacchus"}
\end{minted}

\noindent
Here,
``JSON'' stands for ``JavaScript Object Notation'';
we will learn more about it in \chapref{s:dataman}.

\section{Functions}\label{s:basics-functions}

Functions make it possible for mere mortals to understand programs
by allowing us to think about them one piece at a time.
Here is a function that finds the lowest and highest values in an array:

\begin{minted}{js}
function limits (values) {
  if (!values.length) {
    return [undefined, undefined]
  }
  let low = values[0]
  let high = values[0]
  for (let v of values) {
    if (v < low) low = v
    if (v > high) high = v
  }
  return [low, high]
}
\end{minted}

Its definition consists of the keyword \texttt{function},
its name,
a parameterized list of \gref{g:parameter}{parameters} (which might be empty),
and its body.

The body of the function begins with a test of the thing,
referred to inside the function as \texttt{values},
provided as an argument to the function:
if \texttt{values} has no length---i.e.,
it does not consist of multiple entries---the function returns \texttt{[undefined,undefined]}.
(We will address the rationale behind this behaviour in the exercises.)

\begin{minted}{js}
  if (!values.length) {
    return [undefined, undefined]
  }
\end{minted}

\noindent
If that initial check finds that \texttt{values} does have a length---i.e.,
\texttt{!values.length} returns \texttt{false}---the rest of the function is run.
This involves first initialising two variables,
\texttt{low} and \texttt{high},
with their values set as equal to the first item in \texttt{values}.

\begin{minted}{js}
  let low = values[0]
  let high = values[0]
\end{minted}

In the next stage of the function,
all of the values are iterated over and \texttt{low} and \texttt{high} are
assigned a new value,
equal to that of the next item,
if that value is lower than \texttt{low} or higher than \texttt{high} respectively.

\begin{minted}{js}
  for (let v of values) {
    if (v < low) low = v
    if (v > high) high = v
  }
\end{minted}

Once all of the items in \texttt{values} have been iterated over,
the values of \texttt{low} and \texttt{high} are the minimum and maximum of \texttt{values}.
These are returned as a pair inside an array.

\begin{minted}{js}
  return [low, high]
\end{minted}

\noindent
Note that we can use \texttt{return} to explicitly return a value at any time;
if nothing is returned,
the function's result is \texttt{undefined}.

One oddity of JavaScript is that almost anything can be compared to almost anything else.
Here are a few tests that demonstrate this:

\begin{minted}{js}
const allTests = [
  [],
  [9],
  [3, 30, 300],
  ['apple', 'Grapefruit', 'banana'],
  [3, 'apple', ['sub-array']]
]
for (let test of allTests) {
  console.log(`limits of ${test} are ${limits(test)}`)
}
\end{minted}

\begin{minted}{text}
limits of  are ,
limits of 9 are 9,9
limits of 3,30,300 are 3,300
limits of apple,Grapefruit,banana are Grapefruit,banana
limits of 3,apple,sub-array are 3,3
\end{minted}

Programmers generally don't write functions this way any longer,
since it interacts in odd ways with other features of the language;
\secref{s:legacy-prototypes} explains why and how in more detail.
Instead,
most programmers now write \gref{g:fat-arrow}{fat arrow functions}
consisting of a parameter list,
the \texttt{=\textgreater{}} symbol,
and a body.
Fat arrow functions don't have names,
so the function must be assigned that to a constant or variable for later use:

\begin{minted}{js}
const limits = (values) => {
  if (!values.length) {
    return [undefined, undefined]
  }
  let low = values[0]
  let high = values[0]
  for (let v of values) {
    if (v < low) low = v
    if (v > high) high = v
  }
  return [low, high]
}
\end{minted}

No matter how functions are defined,
each one is a \gref{g:scope}{scope},
which means its parameters and any variables created inside it are \gref{g:local-variable}{local} to the function.
We will discuss scope in more detail in \chapref{s:callbacks}.

\begin{aside}{Stuck in the Past}
  Why did JavaScript introduce another syntax
  rather than fixing the behavior of those defined with \texttt{function}?
  The twin answers are that changes would break legacy programs that rely on the old behavior,
  and that the language's developers wanted to make it really easy to define little functions.
  Here and elsewhere,
  we will see how a language's history and the way it is used shape its evolution.
\end{aside}

\section{Modules}\label{s:basics-modules}

As our programs grow larger,
we will want to put code in multiple files.
The unavoidable bad news is that JavaScript has several module systems:
Node still uses one called CommonJS,
but is converting to the modern standard called ES6,
so what we use on the command line is different from what we use in the browser (for now).

\begin{aside}{Ee Ess}
  JavaScript's official name is ECMAScript,
  though only people who use the word ``datum'' in everyday conversation ever call it that.
  Successive versions of the language are therefore known as ES5, ES6, and so on,
  except when they're referred to as (for example) ES2018.
\end{aside}

Since we're going to build command-line programs before doing anything in the browser,
we will introduce Node's module system first.
We start by putting this code in a file called \texttt{utilities.js}:

\begin{minted}{js}
DEFAULT_BOUND = 3

const clip = (values, bound = DEFAULT_BOUND) => {
  let result = []
  for (let v of values) {
    if (v <= bound) {
      result.push(v)
    }
  }
  return result
}

module.exports = {
  clip: clip
}
\end{minted}

The function definition is straightforward;
as you may have guessed, \texttt{bound\ =\ DEFAULT\_BOUND} sets a default value for that parameter
so that \texttt{clip} can be called with just an array.
You may also have guessed that \texttt{Array.push} appends a value to the end of an array;
if you didn't,
well,
now you know.

What's more important is assigning an object to \texttt{module.exports}.
Only those things named in this object are visible to the outside world,
so \texttt{DEFAULT\_BOUND} won't be.
Remember,
keys that are simple names don't have to be quoted,
so \texttt{clip:\ clip} means ``associate a reference to the function \texttt{clip} with the string key \texttt{"clip"}.

To use our newly-defined module we must \texttt{require} it.
For example,
we can put this in \texttt{application.js}:

\begin{minted}{js}
const utilities = require('./utilities')

const data = [-1, 5, 3, 0, 10]
console.log(`clip(${data}) -> ${utilities.clip(data)}`)
console.log(`clip(${data}, 5) -> ${utilities.clip(data, 5)}`)
\end{minted}

\begin{minted}{text}
clip(-1,5,3,0,10) -> -1,3,0
clip(-1,5,3,0,10, 5) -> -1,5,3,0
\end{minted}

\noindent
\texttt{require} returns the object that was assigned to \texttt{module.exports},
so if we have assigned its result to a variable called \texttt{utilities},
we must then call our function as \texttt{utilities.clip}.
We use a relative path starting with \texttt{./} or \texttt{../} to import local files;
paths that start with names are taken from installed Node libraries.

\figpdf{figures/basics-require.pdf}{How Require Works}{f:basics-require}

\section{Exercises}\label{s:basics-exercises}

\exercise{Typeof}

What kind of thing is \texttt{typeof}?
Is it an expression?
A function?
Something else?
(You might notice that \texttt{typeof\ typeof} is syntactically invalid.
In such circumstances,
an internet search engine is your friend,
as is the Mozilla Developer Network (MDN) JavaScript reference
at \url{https://developer.mozilla.org/en-US/docs/Web/JavaScript/Reference}.

\exercise{Fill in the Blanks}

Answer these questions about the program below:

\begin{enumerate}
\item
  What does \texttt{Array.push} do?
\item
  How does a \texttt{while} loop work?
\item
  What does \texttt{+=} do?
\item
  What does \texttt{Array.reverse} do, and what does it return?
\end{enumerate}

\begin{minted}{js}
let current = 0
let table = []
while (current < 5) {
  const entry = `square of ${current} is ${current * current}`
  table.push(entry)
  current += 1
}
table.reverse()
for (let line of table) {
  console.log(line)
}
\end{minted}

\begin{minted}{text}
square of 4 is 16
square of 3 is 9
square of 2 is 4
square of 1 is 1
square of 0 is 0
\end{minted}

\exercise{What Is Truth?}

Write a function called \texttt{isTruthy} that returns \texttt{true} for everything that JavaScript considers truthy,
and \texttt{false} for everything it considers \texttt{falsy} \emph{except} empty arrays:
\texttt{isTruthy} should return \texttt{false} for those.

\exercise{The Shape of Things to Come}

We wrote the example function, \texttt{limits},
above to return \texttt{[undefined,undefined]} if a variable
with no length is fed into it.
What is the advantage of doing this as opposed to
returning \texttt{undefined} only?

\exercise{Combining Different Types}

What output would you expect from the code below?
Try running it and see whether the results match your expectations.
What are the implications of this behavior when working with real-world data?

\begin{minted}{js}
const first = [3, 7, 8, 9, 1]
console.log(`aggregating ${first}`)
let total = 0
for (let d of first) {
  total += d
}
console.log(total)

const second = [0, 3, -1, NaN, 8]
console.log(`aggregating ${second}`)
total = 0
for (let d of second) {
  total += d
}
console.log(total)
\end{minted}

\exercise{What Does This Do?}

Explain what is happening in the assignment statement that creates the constant \texttt{creature}.

\begin{minted}{js}
const genus = 'Callithrix'
const species = 'Jacchus'
const creature = {genus, species}
console.log(creature)
\end{minted}

\begin{minted}{text}
{ genus: 'Callithrix', species: 'Jacchus' }
\end{minted}

\exercise{Destructuring Assignment}

When this short program runs:

\begin{minted}{js}
const creature = {
  genus: 'Callithrix',
  species: 'Jacchus'
}
const {genus, species} = creature
console.log(`genus is ${genus}`)
console.log(`species is ${species}`)
\end{minted}

\noindent
it produces the output:

\begin{minted}{text}
genus is Callithrix
species is Jacchus
\end{minted}

\noindent
but when this program runs:

\begin{minted}{js}
const creature = {
  first: 'Callithrix',
  second: 'Jacchus'
}
const {genus, species} = creature
console.log(`genus is ${genus}`)
console.log(`species is ${species}`)
\end{minted}

\noindent
it produces:

\begin{minted}{text}
genus is undefined
species is undefined
\end{minted}

\begin{enumerate}
\item
  What is the difference between these two programs?
\item
  How does \gref{g:destructuring}{destructuring assignment} work in general?
\item
  How can we use this technique to rewrite the \texttt{require} statement in \texttt{src/basics/import.js}
  so that \texttt{clip} can be called directly as \texttt{clip(...)} rather than as \texttt{utilities.clip(...)}?
\end{enumerate}

\exercise{Return To Me, For My Heart Wants You Only}

What output would you see in the console if you ran this code?

\begin{minted}{js}
const verbose_sum = (first, second) => {
  console.log(`Going to add ${first} to ${second}`)
  let total = first + second
  return total
  console.log(`Finished summing`)
}

var result = verbose_sum(3, 6)
console.log(result)
\end{minted}

\begin{enumerate}
\item
  \texttt{Going\ to\ add\ \$\{first\}\ to\ \$\{second\}}\\
  \texttt{9}
\item
  \texttt{Going\ to\ add\ 3\ to\ 6}\\
  \texttt{9}\\
  \texttt{Finished\ summing}
\item
  \texttt{Going\ to\ add\ 3\ to\ 6}\\
  \texttt{9}
\item
  \texttt{Going\ to\ add\ 3\ to\ 6}\\
  \texttt{36}
\end{enumerate}

\section*{Key Points}

\begin{itemize}
\item
  Use \texttt{console.log} to print messages.
\item
  Use dotted notation \texttt{X.Y} to get part \texttt{Y} of object \texttt{X}.
\item
  Basic data types are Booleans, numbers, and character strings.
\item
  Arrays store multiple values in order.
\item
  The special values \texttt{null} and \texttt{undefined} mean `no value' and `does not exist'.
\item
  Define constants with \texttt{const} and variables with \texttt{let}.
\item
  \texttt{typeof} returns the type of a value.
\item
  \texttt{for\ (let\ variable\ of\ collection)\ \{...\}} iterates through the values in an array.
\item
  \texttt{if\ (condition)\ \{...\}\ else\ \{...\}} conditionally executes some code.
\item
  \texttt{false}, 0, the empty string, \texttt{null}, and \texttt{undefined} are false; everything else is true.
\item
  Use back quotes and \texttt{\$\{...\}} to interpolate values into strings.
\item
  An object is a collection of name/value pairs written in \texttt{\{...\}}.
\item
  \texttt{object{[}key{]}} or \texttt{object.key} gets a value from an object.
\item
  Functions are objects that can be assigned to variables, stored in lists, etc.
\item
  \texttt{function\ name(...parameters...)\ \{...body...\}} is the old way to define a function.
\item
  \texttt{name\ =\ (...parameters...)\ =\textgreater{}\ \{...body...\}} is the new way to define a function.
\item
  Use \texttt{return} inside a function body to return a value at any point.
\item
  Use modules to divide code between multiple files for re-use.
\item
  Assign to \texttt{module.exports} to specify what a module exports.
\item
  \texttt{require(...path...)} imports a module.
\item
  Paths beginning with `.' or `/' are imported locally, but paths without `.' or `/' look in the library.
\end{itemize}

