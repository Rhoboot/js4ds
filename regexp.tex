\chapter{Regular Expressions}\label{s:regexp}

A \gref{g:regular-expression}{regular expression} is a pattern for matching text.
Most languages implement them in libraries,
but they are built into JavaScript,
and are written using \texttt{/} before and after a string rather than single or double quotes.

\begin{itemize}

\item
  Letters and digits match themselves,
  so the regular expression \texttt{/enjoy/} matches the word ``enjoy'' wherever it appears in a string.

\item
  A dot matches any character,
  so \texttt{/../} matches any two consecutives characters
  and \texttt{/en..y/} matches ``enjoy'', ``gently'', and ``brightens your day''.

\item
  The asterisk \texttt{*} means ``match zero or more occurrences of what comes immediately before'',
  so \texttt{en*} matches ``ten'' and ``penny''.
  It also matches ``feet'' (which has an ``e'' followed by zero occurrences of ``n'').

\item
  The plus sign \texttt{+} means ``match \emph{one} or more occurrences of what comes immediately before'',
  so \texttt{en+} still matches ``ten'' and ``penny'' but doesn't match ``feet''
  (because there isn't an ``n'').

\item
  Parentheses create groupes just as they do in mathematics,
  so \texttt{(an)+} matches ``banana'' but not ``annual''.

\item
  The pipe character \texttt{|} means ``either/or'',
  so \texttt{b|c} matches either a single ``b'' or a single ``c'',
  and \texttt{(either)|(or)} matches either ``either'' or ``or''.
  (The parentheses are necessary because \texttt{either|or} matches ``eitherr'' or ``eitheor''.)

\item
  The shorthand notation \texttt{[a-z]} means ``all the characters in a range'',
  and is easier to write and read than \texttt{a|b|c|{\ldots}|y|z}.
  
\item
  If we want to put a special character like \texttt{.}, \texttt{*}, \texttt{+}, or \texttt{|}
  in a regular expression,
  we have to \gref{g:escape-sequence}{escape} it with a backslash \texttt{\textbackslash}.
  This means that \texttt{/stop{\textbackslash}./} only matches ``stop.'',
  while \texttt{stop.} matches ``stops'' as well.
  
\end{itemize}

FIXME: add a couple of simple regular expression examples here.

\figpdf{figures/FIXME.png}{Regular Expressions}{f:regexp-fsa} % simple state machine for regexp matching

Regular expressions can match an intimidating number of other patterns,
but are fairly easy to use in programs.
Like strings and arrays,
they are objects with methods:
if \texttt{pattern} is a regular expression,
then \texttt{string.test(pattern)} returns \texttt{true} if the pattern matches the string
and \texttt{false} if it does not,
while \texttt{string.match(pattern)} returns an array of matching substrings.
If we add the modifier ``g'' after the closing slash of the regular expression to make it ``global'',
then \texttt{string.match(pattern)} returns \emph{all} of the matching substrings:

\begin{minted}{js}
Tests = [
  'Jamie: james@geneinfo.org',
  'Zara: zetsure@bio123.edu',
  'Hong and Andrzej: hchui@euphoric.edu and aszego@euphoric.edu'
]

const pattern = /[a-z]+@[a-z]+\.[a-z]+/g

console.log(`pattern is ${pattern}`)
for (let test of Tests) {
  console.log(`\ntested against ${test}`)
  const matches = test.match(pattern)
  if (matches === null) {
    console.log('-no matches-')
  }
  else {
    for (let m of matches) {
      console.log(m)
    }
  }
}
\end{minted}

\begin{minted}{text}
pattern is /[a-z]+@[a-z]+\.[a-z]+/g

tested against Jamie: james@geneinfo.org
james@geneinfo.org

tested against Zara: zetsure@bio123.edu
-no matches-

tested against Hong and Andrzej: hchui@euphoric.edu and aszego@euphoric.edu
hchui@euphoric.edu
aszego@euphoric.edu
\end{minted}

As powerful as they are,
there are things that \hreffoot{https://stackoverflow.com/questions/1732348/regex-match-open-tags-except-xhtml-self-contained-tags/1732454\#1732454}{regular expressions can't do}.
When it comes to pulling information out of text,
though,
they are easier to use and more efficient than long chains of substring tests.
They can also be used to replace substrings and to split strings into pieces:
please see \hreffoot{https://developer.mozilla.org/en-US/docs/Web/JavaScript/Guide/Regular\_Expressions}{the documentation} for more information.
